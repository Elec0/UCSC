% Headers
\documentclass[12pt]{article}
\usepackage{graphics}
\usepackage{verbatim}
\usepackage{hyperref}
\usepackage{amsmath}
\usepackage{enumerate}
\usepackage{xcolor, graphicx}
\usepackage{booktabs}
\usepackage{array}
\usepackage{parskip}
\usepackage[margin=2.5cm]{geometry}
\usepackage{listings}
\usepackage{mathtools}
\DeclarePairedDelimiter\ceil{\lceil}{\rceil}
\DeclarePairedDelimiter\floor{\lfloor}{\rfloor}

\title{\textbf{CMPS 111 Operating Systems\\ Winter 2018, Homework \#1}}
\date{}
\author{Aaron Steele, atsteele@ucsc.edu}

\begin{document}

  %------------------------------------------------------------------------
  % Page 1
  %------------------------------------------------------------------------

  \maketitle
  
  \section*{Question 1}
  
  \begin{itemize}
  	\item 1940s-1950s
  	\begin{itemize}
  		\item The era prior to operating systems. Computers were programmed with assembly language directly, sometimes with plugboards.
  	\end{itemize}
  	\item 1950s-1960s
  	\begin{itemize}
  		\item The first operating system used for real work was the GM-NAA I/O, created by the General Motors' Research Division in 1956
  		\item Most other operating systems were also written by consumers at this point
  	\end{itemize}
  	\item 1960s-1965
  	\begin{itemize}
  		\item IBM wanted to make a single operating system for their OS/360 hardware, using the same instruction architecture.
  		\item This was the era of mainframe operating systems, when many different companies creates a whole slew of new operating systems that all worked totally differently.
  	\end{itemize}
	\item Late 1960s
	\begin{itemize}
	  	\item The Unix operating system was created in AT\&T's Bell Laboratories. Unix was programmed in C, which meant that when that language was ported to a new CPU architecture, the entire OS could also be ported. Because of this, it started gaining a large amount of popularity.
	\end{itemize}
	\item 1970s - Microcomputers: small computers with 8-bit processors, like the Intel 8080, these were some of the first affordable computers on the market for hobbyists
	\begin{itemize}
		\item Digital Research's CP/M-80 for the 8080/8085/Z-80 CPUs. Was mainly based on the PDP-11 architecture.
		\item Microsoft's first operating system MSDOS/MIDAS was also developed on the PDP-11 architecture.
		\item Apple DOS was released in 1978.
	\end{itemize}
 	\item 1980s
 	\begin{itemize}
 		\item MS-DOS was released in 1981
 		\item Mac OS was released in 1984
 		\item Windows 1.0 was released in 1985
 		\item Version 8 of Unix was released in 1985
 	\end{itemize}
	\item These four operating systems started to dominate the market, especially in the late 1980s and into the 1990s, when they became the go-to operating systems for a majority of manufactured computers.
  \end{itemize}


  \section*{Question 2}
  
  There are a total of four processes created by running this code.
  First, P0 is the original process. P0 runs the first fork() and creates P1. P1 now runs the second fork() and creates P2. P2 runs exit(1) and quits.\\
  After P0 has ran the first fork(), it runs the second and creates P3, which then runs exit(1) and quits.
  
  So, the program creates four processes, but all of these processes might not be running at the same time, since it's easily possible that P0 might have exited by the time P2 is created, depending on the machine.
  
  
  \section*{Question 3}
  A 32-bit unsigned integer has a maximum value of 4,294,967,295 $(2^{32}-1)$. Converting the max value of a 32-bit unsigned integer to a timestamp gives us: Sunday, February 7, 2106 6:28:15 AM. This would give us 88 years before the value overflows. 
  
  I would argue that is not enough time and that a 64-bit value should be used, which would give a much larger timeframe of 2,874 years.
  
  \section*{Question 4}  
  A web server that has a large amount of traffic has a number of problems, but the primary one, to my mind, is being able to serve the amount of data requested to each user from either the database or it's local fileserver. These are all read operations, which means that no locking is required and multiple threads can access the data without worrying about corruption or things of that nature.
  
  Since the data can be accessed by multiple threads at once, it becomes fairly simple to load the data into shared memory and allow the threads to all read from it. This allows us to scale up the number of threads to meet the demand of traffic very easily, whereas using a single thread, or just a few threads, would be a much more challenging problem to serve the data to that many users. 
  
  This type of structure can be created using a thread pool with a circular bounded buffer. This also allows us to change the data that is being served to users, which we need to be able to do if we ever want to update the web server without fully taking the service down.   
  
  
  \section*{Question 5}
  
  \subsection*{Part A}
  Given $U = 1-p^n$, we know U, and p, n is our unknown.
  $$U = 0.99$$
  $$p=0.60$$
  $$0.99 = 1-0.60^n$$
  $$0.99-1=0.60^n$$
  $$\log 0.01 = (\log 0.60)^n$$
  $$\log0.01=n\log0.60$$
  $$\floor{\frac{\log0.01}{\log0.60}} = n$$
  $$9 = n$$
  
  Therefore, there are 9 processes running at 99\% CPU utilization. So, $16-9$MB = 7MB of memory left free.
  
  \subsection*{Part B}
  3MB per process. So, $\floor{16/3} = 5$ maximum processes.
  $$U = 1 - 0.60^5$$
  $$U = 0.92$$
  
  Therefore, the CPU has 92\% utilization in this scenario.
  
  
  
\end{document}