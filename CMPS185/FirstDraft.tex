\documentclass[12pt,twocolumn]{article}
\usepackage{url}
\usepackage{listings}
\usepackage{textcomp}
\usepackage{amsmath}
\usepackage[T1]{fontenc} % Turns quotes straight
\bibliographystyle{ieeetr}

\newcommand\blfootnote[1]{%
	\begingroup
	\renewcommand\thefootnote{}\footnote{#1}%
	\addtocounter{footnote}{-1}%
	\endgroup
}


\title{Quality-controlling real-time crowdsourced data \\ \large First Draft}

\author{Aaron Steele}
\date{}


\begin{document}
	\maketitle
	
	\blfootnote{Conference: }
	\blfootnote{Papers: \cite{UB-QC-CC}\cite{QC-CC}}
	
	% Outline
	% I. Quality controlling normal crowdsourced data [1st paper]
	%	1. Web-based crowdsourcing
	%	2. QC of standard crowdsourcing
	%	3. Taxonomy of quality in crowdsourcing systems
	%		1. Worker: Reputation & Expertise
	%		2. Task: Definition, User Interface, Granluarity, Compensation Policy
	%	4. Quality control approaches [table] (Contribuitor evaulation?)
	% II. Intro to real-time QC
	%	1. Definition of UB QC
	%	2. Problems with dynamic crowds and UB data
	%	3. Participatory sensing
	% III. The credibility-weight and proposed approach
	%	1. Using mobility patterns
	%	2. Defining POIs via tuples
	%	3. Regularity function (how regularly a user visits a place)
	%	4. Trustworthiness score, or reputation
	%	5. Final formula and explanation
	
	\section*{Abstract}
	    TODO
	\section*{Introduction}
	    Crowdsourcing has become a much more popular way of both getting and processing data in recent years. In processing data, crowdsourcing is primarily helpful for tasks which are easy for humans to do but hard to machines. %\cite
	    Getting data, on the other hand, is a very different process. With the advent of the ubiquity of smartphones, it has become much easier to gather various types of data from the crowd, as it were.
	    
	    Quality controlling crowdsourced processing is, in a lot of ways, easier than doing the same with crowdsourced data. We have created a taxonomy of quality in crowdsourced systems. There are various approaches to quality controlling the processed data, but we believe the most relevant one for our purposes is Contributor Evaluation, which is where a contribution is assessed based on the quality of the contributor's previous contributions. 
	    %Recreate and include the taxonomy diagram?
	    
	\section*{Body}
	    Thanks to the rise of smartphones being abundant, a new type of crowdsourcing has been created. Ubiquitous crowdsourcing is smartphone owners contributing data about their outside world, such as GPS location, or ambient noise level. 
	    
	\section*{Related Works}
	    
	    
	\section*{Conclusion}
	    TODO
	
	\bibliography{FirstDraft}
\end{document}